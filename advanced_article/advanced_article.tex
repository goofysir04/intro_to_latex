% use a standard article class, but make the text 10pt
\documentclass[10pt]{article}

% specify the title, authors, etc.
\title{A review of LaTeX for generating journal articles}
\author{David P. Larson and Michelle Ferrez}
\date{\today}  % show the current date

% use the AMS LaTeX packages for better equations, math symbols, etc.
\usepackage{amsmath}
\usepackage{amssymb}

% required package for insert graphics
\usepackage{graphicx}



% start the LaTeX document
\begin{document}

% insert the title, authors, etc.
\maketitle

%-----------------------------------------------------------------------------
% ABSTRACT
%-----------------------------------------------------------------------------
\begin{abstract}
    We review some concrete examples of formatting a journal article forv
    submission using LaTeX. Specifically, we cover how to use LaTeX packages
    for improved functionality, adding figures and tables, and citing
    references (including auto-generating the bibliography from a BibTeX file).
\end{abstract}


%-----------------------------------------------------------------------------
% INTRO
%-----------------------------------------------------------------------------
\section{Introduction}


%-----------------------------------------------------------------------------
% EQUATIONS
%-----------------------------------------------------------------------------
\section{Equations}

\subsection{Simple equations}
The standard way to produce one-line equations in LaTeX is as follows:
\begin{equation}
    y = ax + b
\end{equation}

You can group multiple lines of equations together as follows
\begin{eqnarray}
    y = ax + b \\
    z = \alpha x^2 + \beta x + \gamma
\end{eqnarray}

\subsection{Advanced equations}
Most journals however recommend (or sometimes require) you to instead use the
American Mathematical Society (AMS) LaTeX packages for creating advanced
equations. Here's a one line equation using the AMS package
\begin{align}
    y = ax + b
\end{align}

And here is a multi-line equation set
\begin{align}
    y &= ax + b \\                      % the "&" sets where to align the two equations
    z &= \alpha x^2 + \beta x + \gamma  % in this case, it aligns the equal signs
\end{align}

Here's another example: a multi-line derivation of the gradient of a function
\begin{align}
    \nabla f(x) &= \nabla ( \frac{1}{2} x^T A x + b ) \\
                 &= \\
\end{align}

The AMS package also includes functions to produce vectors and matrices:
\begin{align}
    A = \begin{bmatrix}    % bmatrix = bracket matrix, pmatrix = paranthesis matrix, etc.
        A_{1,1} & A_{1,2} & A_{1,3} \\   % the "&" split up the columns
        A_{2,1} & A_{2,2} & A_{2,3} \\
        A_{3,1} & A_{3,2} & A_{3,3} \\
    \end{bmatrix}
\end{align}




%-----------------------------------------------------------------------------
% FIGURES
%-----------------------------------------------------------------------------
\section{Figures}


%-----------------------------------------------------------------------------
% TABLES
%-----------------------------------------------------------------------------
\section{Tables}



%-----------------------------------------------------------------------------
% CITATIONS
%-----------------------------------------------------------------------------
\section{References and Citations}

\subsection{References}
Referring to items within the document.


\subsection{Citations}
Citing sources from a bibliography file.

\cite{Larson2016} is an example journal article that was prepared using LaTeX
and published in the Renewable Energy journal.



%-----------------------------------------------------------------------------
% CONCLUSIONS
%-----------------------------------------------------------------------------
\section{Conclusions}




%-----------------------------------------------------------------------------
% BIBLIOGRAPHY
%-----------------------------------------------------------------------------

% set the bibliography style
% - plain = numbered
%
\bibliographystyle{plain}

% the BibTeX file that has all the citation info
\bibliography{citations.bib}


\end{document}
